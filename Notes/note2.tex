\documentclass[cn,hazy,green,12pt,normal]{elegantnote}

\title{回归分析第三次习题课}
\author{卞泽宇}
\institute{USTC}

\date{\today}

\usepackage{array}

\usepackage{amsmath, amssymb, bm, color, framed, graphicx, hyperref, mathrsfs, fontspec, geometry, extarrows, amsthm}

\DeclareMathOperator{\e}{\!\!\;\mathrm e}
\DeclareMathOperator{\Cov}{Cov}
\DeclareMathOperator{\Var}{Var}
\DeclareMathOperator{\var}{var}
\DeclareMathOperator{\tr}{tr}
\DeclareMathOperator{\diag}{diag}
\newcommand{\p}{\partial}
\renewcommand{\d}{\mathop{}\!\mathrm{d}}
\newcommand{\MR}{\mathbb R}
\newcommand{\MC}{\mathbb C}
\newcommand{\MF}{\mathbb F}
\newcommand{\MZ}{\mathbb Z}
\newcommand{\MN}{\mathbb N}
\newcommand{\MCF}{\mathscr F}
\renewcommand{\Re}{\operatorname{Re}}
\renewcommand{\Im}{\operatorname{Im}}
\renewcommand{\boldsymbol}{\bm}
\renewcommand{\i}{\mathrm i}

\DeclareMathOperator{\Arg}{Arg}
\DeclareMathOperator{\I}{I}
\usepackage{tkz-euclide}
\numberwithin{equation}{section}
\numberwithin{subsection}{section}

\lstset{
    language=R,
    basicstyle=\ttfamily,
    keywordstyle=\color{blue},
    commentstyle=\color{gray},
    frame=single,
    breaklines=true
}

\begin{document}

\maketitle

\tableofcontents

\section{补充内容}
\subsection{约束最小二乘估计、方差分析}
从线性空间分解的角度看约束最小二乘,我们考虑约束条件$A\bm\beta = c$,$A$为$q\times p$的行满秩矩阵。(也可以运用到方差分析中的$H_0$下模型)
其解集为“线性空间+平移”,因此我们先考虑约束条件$A\bm\beta=0$。
\begin{theorem}
    设$X \in \MR^{n\times p}$列满秩,$A \in \MR^{q\times p}$行满秩,$V=C(X),V_0=\left\{X\bm \beta | A\bm \beta = 0  \right\}$

    则\[V_0 = V \cap C(X(X^TX)^{-1}A^T)^{\perp},\quad P_V = P_{V_0}+P_{X(X^TX)^{-1}A^T}\]
\end{theorem}

\begin{proof}
    左包含$\subseteq$:
\[
    \forall \bm u \in V_0, \bm u = X\bm \beta_u \in V,\quad A
    \beta_u=0,\quad
    \forall \bm v = X(X^TX)^{-1}A^Tw\in C(X(X^TX)^{-1}A^T),
    \]
    
    $\langle u,v\rangle=\langle X\bm \beta_u,X(X^TX)^{-1}A^Tw \rangle=
    \langle \beta_u,A^Tw\rangle=\langle A
    \beta_u,w\rangle=0$,
    因此$u\in C(X(X^TX)^{-1}A^T)^{\perp}$,进而左包含。
    
\noindent 右包含$\supseteq$:
\[\forall u \in V\cap C(X(X^TX)^{-1}A^T)^{\perp}, \quad u=X\bm \beta_u ,\quad \forall v\in \MR^{q}:\]
$0 = \langle u,X(X^TX)^{-1}A^Tv\rangle =\langle X\bm \beta_u,X(X^TX)^{-1}A^Tv \rangle =
\langle \beta_u,A^Tv\rangle =\langle A\beta_u,v\rangle$。取$v = A\beta_u$得到$A\beta_u=0$,故$u\in V_0$

进而由于$C(X(X^TX)^{-1}A^T) \subseteq V$,故$V = V_0 \oplus C(X(X^TX)^{-1}A^T)$,从而得到投影算子分解。
\end{proof}

\begin{corollary}
    \begin{enumerate}
        \item 满足条件$A\bm\beta=0$的约束最小二乘估计为
        \[\hat{\bm\beta}_{con}=\hat{\bm\beta} - (X^TX)^{-1}A^T(A(X^TX)^{-1}A^T)^{-1}A\hat{\bm\beta}\]
        \item 满足条件$A\bm\beta=b$的约束最小二乘估计为
        \[\hat{\bm\beta}_{con}=\hat{\bm\beta} - (X^TX)^{-1}A^T(A(X^TX)^{-1}A^T)^{-1}(A\hat{\bm\beta}-b)\]
    \end{enumerate}
\end{corollary}

\begin{proof}
(1)我们有
\begin{align*}
    X\hat{\bm\beta}_{con}=P_{V_0}\bm y &= P_{V}\bm y - P_{X(X^TX)^{-1}A^T}\bm y\\
    &=X\hat{\bm\beta}-(X(X^TX)^{-1}A^T)(A(X^TX)^{-1}A^T)^{-1}A(X^TX)^{-1}X^T\bm y \\
    &=X\hat{\bm\beta}-(X(X^TX)^{-1}A^T)(A(X^TX)^{-1}A^T)^{-1}A\hat{\bm\beta}
\end{align*}  
从而两侧乘$X$左逆($X$列满秩)得到结论。

\noindent (2)设延正交补空间平移向量为$X(X^TX)^{-1}A^T u$则
\[
\begin{cases}
    X\hat{\bm\beta}_{con}&=P_{V_0}\bm y+X(X^TX)^{-1}A^T u\\
    b&=A\hat{\bm\beta}_{con}=A(\hat{\bm\beta} - (X^TX)^{-1}A^T(A(X^TX)^{-1}A^T)^{-1}A\hat{\bm\beta}+(X^TX)^{-1}A^Tu)
\end{cases}
\]

故$u=(A(X^TX)^{-1}A^T)^{-1}b$
代入即得
\end{proof}
\subsection[测定系数及其多元推广]{$R^2$及其多元推广}
我们知道$R^2$是对回归模型拟合优度的一种度量,刻画因变量能被自变量线性解释部分所占的比例($RSS_{reg}/RSS$)。

先从总体模型出发,在第一节习题课中,我们知道决定系数$\phi:=\dfrac{\var(\bm x^T b)}{\var (y)}=\dfrac{\Sigma_{y\bm x}\Sigma_{\bm x\bm x}^{-1}\Sigma_{\bm x y}}{\Sigma_{yy}}$。刻画了随机变量$y$被随机变量$x\in \MR^n$的分量线性组合解释的程度。将总体模型中的参数替换为估计值,即得到$R^2=\dfrac{S_{y\bm x}S_{\bm x\bm x}^{-1}S_{\bm x y}}{S_{yy}}=\dfrac{(SYX) (SXX)^{-1}(SXY)}{SYY}$

\begin{definition}[典则相关系数 Canonical Correlation Coefficient]
    设随机变量$\bm y \in \MR^m,\bm x \in \MR^n$考虑半正定阵\[\Phi =\Sigma_{\bm y\bm y}^{-\frac{1}{2}}(\Sigma_{\bm y\bm x}\Sigma_{\bm x\bm x}^{-1}\Sigma_{\bm x\bm y})\Sigma_{\bm y \bm y}^{-\frac{1}{2}}\]
    其最大特征根的平方根$\sqrt{\lambda_{max}(\Phi)}$称为第一典则相关系数。
\end{definition}

\begin{remark}
    (1)m=1时,即为决定系数$\phi$,这是在回归分析中重点讨论的内容。
    
    \noindent (2)同样刻画了$\bm x$和$\bm y$的相关程度,见定理\ref{relation}。
\end{remark}

\begin{theorem}
    设A和B为$n\times n$对称矩阵,定义$A\leq B:$若$A-B\leq 0$(半负定)。设$\lambda(A)$表示A的任意特征值,则:

    \begin{enumerate}
        \item 若$A\leq I$则$\lambda(A)\leq 1$
        \item $0\leq \Phi \leq I$
        \item $0\leq \lambda(\Phi)\leq 1$
    \end{enumerate}
\end{theorem}

\begin{proof}
   注意到$\Sigma_{\bm y\bm y \cdot \bm x}\geq 0$,其余自证不难。
\end{proof}

\begin{theorem}\label{relation}
    设随机变量$\bm y \in \MR^m,\bm x \in \MR^n$,常向量$\bm a,\bm b$,则
    \[
    \max_{\substack{0\neq \bm a \in \MR^n\\0\neq \bm b \in \MR^m}} \rho(\bm a^T \bm x,\bm b^T \bm y)=\sqrt{\lambda_{max}(\Phi)}
    \]
\end{theorem}

\begin{proof}
\begin{align*}
        |\rho(\bm a^T \bm x,\bm b^T \bm y)|&=\dfrac{|\Cov(\bm a^T \bm x,\bm b^T \bm y)|}{\sqrt{\Var(\bm a^T \bm x)}\sqrt{\Var(\bm b^T \bm y)}}=\dfrac{|\bm a^T\Sigma_{\bm x \bm y}\bm b|}{\sqrt{\bm a^T \Sigma_{\bm x \bm x}\bm a \sqrt{\bm b^T \Sigma_{\bm y\bm y}\bm b}}}
\end{align*}
令$\bm u = \Sigma_{\bm x \bm x}^{\frac{1}{2}}\bm a,\quad \bm v = \Sigma_{\bm y \bm y}^{\frac{1}{2}}\bm b$
\begin{align*}
    LHS = \dfrac{|\bm u ^T(\Sigma_{\bm x\bm x}^{-\frac{1}{2}}\Sigma_{\bm x\bm y}\Sigma_{\bm y\bm y}^{-\frac{1}{2}}\bm v)|}{\sqrt{\bm u^T\bm u}\sqrt{\bm v^T\bm v}}&\leq  \dfrac{\sqrt{\bm u^T\bm u}\sqrt{(\Sigma_{\bm x\bm x}^{-\frac{1}{2}}\Sigma_{\bm x\bm y}\Sigma_{\bm y\bm y}^{-\frac{1}{2}}\bm v)^T(\Sigma_{\bm x\bm x}^{-\frac{1}{2}}\Sigma_{\bm x\bm y}\Sigma_{\bm y\bm y}^{-\frac{1}{2}}\bm v)}}{\sqrt{\bm u^T\bm u}\sqrt{\bm v^T\bm v}}\\
    &=\dfrac{\sqrt{\bm v^T \Phi \bm v}}{\sqrt{\bm v^T\bm v}}\leq \sqrt{\lambda_{max}(\Phi)}
\end{align*}
两次放缩分别用到了Cauchy不等式和矩阵二范数性质。利用两次取等条件可以确定$\bm a,\bm b$(不妨设两者模长为1)。
\end{proof}

类比到线性模型中,我们有
\begin{theorem}设有截距线性回归模型$\bm y=X\bm \beta +e$满足Gauss-Markov假设。则
    \[R^2=\max_{\bm u \in C(X)}(r_{\bm u,\bm y})^2=(r_{\bm y,\hat{\bm y}})^2=\dfrac{s_{y\bm x}s_{\bm x\bm x}^{-}s_{\bm x y}}{s_{yy}}\]
\end{theorem}
\begin{proof}
    参考PPT定理1.2.6证明。
\end{proof}


\end{document}